\section{An Example of Rigorous Benchmark Reporting}
\label{sec:app-example}

In this section, we present a modified reporting example based on \birdbench to 
demonstrate benchmark reporting that fulfills all the criteria outlined in 
Figure \ref{fig:report}. \birdbench is a benchmark for evaluating agents' 
capability to translate a natural language query to a SQL query.

\minihead{R.1} Is fully or at least partially open-sourced.

\textbf{Example}: We released the training and validation dataset of \birdbench 
at \url{https://bird-bench.github.io/}.


\minihead{R.2} Offers an open-source evaluation harness for users.

\textbf{Example}: We released the harness to evaluation agents on \birdbench at 
\url{https://github.com/AlibabaResearch/DAMO-ConvAI/tree/main/bird}.

\minihead{R.3} Includes measures to prevent data contamination, such as a
private, held-out test set.

\textbf{Example}: We keep a private held-out test set to avoid potential data 
contamination. Request to evaluate agents on this test set can be submitted at 
\url{https://bird-bench.github.io/}.

\minihead{R.4} Includes measures or plans to consistently update challenges over
time to avoid overfitting.

\textbf{Example}: We plan to consistently update the database and natural 
language queries to reflect the real-world queries and avoid overfitting. Our 
updates will be available at \url{https://bird-bench.github.io/}.

\minihead{R.5} Clearly states the relationship between the agent capabilities it
aims to evaluate and the constructs or outcomes it measures.

\textbf{Example}: \birdbench evaluates agents' capabilities to serve as a 
database interface to translate natural language queries into executable SQL 
queries. To achieve that, \birdbench provides agents with a natural language
query, the database schema, and SQL-related domain knowledge, and challenges 
agents to write a SQL query that can be executed to return correct answers.

\minihead{R.6} Clearly states the evaluation subjective of the benchmark (e.g., a
model or an agent framework).

\textbf{Example}: \birdbench is designed to evaluate the capability of ML models
as well as the performance of agent frameworks.

\minihead{R.7} Describes steps taken to prevent, identify, and correct flaws.

\textbf{Example}: We identify that evaluating generated SQL queries using 
execution results have two limitations. First, tasks requiring \texttt{LIMIT}
queries and containing ties in the data may lead to non-deterministic execution
results. Second, manually annotated ground-truth queries may contain errors. To 
understand and mitigate these errors, we randomly sample 500 tasks to perform
an additional phase of verification. After verifying queries, we found 11.65\% 
of ground-truth queries are incorrect.\footnote[4]{We used results by 
\citet{bird-dev-errors}.}


\minihead{R.8} Includes qualitative discussions of the potential impact of 
unavoidable flaws.

\textbf{Example}: The identified incorrect ground-truth queries and potentially
more incorrect ground-truth queries in the test dataset can lead to estimation
errors of the agent performance and incorrect rankings of agents.

\minihead{R.9} Includes quantitative analysis to assess the impact of unavoidable
flaws (e.g., noise of ground truth).

\textbf{Example}: We build our quantitative analysis based on the normality 
assumption. Specifically, suppose the number of data in the test set $N$ is 
large enough such that the true success rate ($p$) of an agent follows a normal 
distribution with mean $\mu$ and standard deviation $\sigma$. Given the ground 
truth's incorrectness rate of $e$ and the estimated agent success rate $p_0$ 
(based on the imperfect ground truth), $\mu$ and $\sigma$ are calculated as
\begin{equation*}
    \mu = e + (1-2e)p_0; \quad
    \sigma^2 = \mu(1-\mu) = \left(e + (1-2e)p_0\right) \left(1- e - (1-2e)p_0\right)
\end{equation*}
Hence, based on the normality assumption, we can derive a two-sided confidence 
interval with confidence $\alpha$ for $p$ as follows:
\begin{equation}
    \mathbb{P}\left[ \mu - 1.96 \times \frac{\sigma}{\sqrt{N}} \le p \le \mu + 1.96 \times \frac{\sigma}{\sqrt{N}} \right] \ge 95\%
\end{equation}

Finally, based on the plug-in estimate (11.65\%) for the ground truth's 
incorrectness rate, we calculate the confidence interval for the agents' 
performance in Table \ref{tab:bird-data}.

\minihead{R.10} Reports metrics about statistical significance, such as
confidence intervals.

\textbf{Example}: In additional to accuracy estimate, we also calculate 
confidence intervals for each model in Table \ref{tab:bird-data}.

\minihead{R.11} Provides guidance on interpreting results with eval flaws.

\textbf{Example}: Given the potential flaws in \birdbench, we do not recommend 
users to rely on the success rate alone for decision-making or selecting models.
Instead, we suggest using the confidence interval of the success rate as a 
reference.

\minihead{R.12} Reports results of non-AI baselines (e.g., human experts).

\textbf{Example}: We measured the performance of a SQL expert on \birdbench, 
obtaining a success rate of 92.96\%. 

\minihead{R.13} Reports results of trivial agents (e.g., one that does nothing).

\textbf{Example}: We performed sanity check on our evaluation harness by 
measuring the performance of a trivial agent that does nothing. We find that the 
trivial agent achieves 0\% success rate, confirming the rigor of our evaluation
implementation.

{\scriptsize
\begin{longtblr}[
    caption = {Modified Leaderboards of \birdbench \cite{li2023can} with Confidence Intervals.},
    label = {tab:bird-data} ]{colspec={Q[c]Q[r]Q[r]Q[r]Q[r]},row{1} = {font=\bfseries}, row{odd[2]} = {bg=gray!25}}% 
\toprule
        Method & Dev. Accuracy (\%) & Confidence Interval &  Original Rank & Possible Rank  \\ 
\midrule
CHASE-SQL + Gemini&74.9&[66.8, 71.4]&1&1-13\\
Contextual-SQL&73.5&[65.7, 70.4]&2&1-16\\
XiYan-SQL&73.3&[65.6, 70.2]&3&1-18\\
ExSL + granite-34b-code&72.4&[64.9, 69.6]&4&1-22\\
Reasoning-SQL-14B&72.3&[64.7, 69.4]&5&1-22\\
Insights AI&72.2&[64.6, 69.4]&6&1-22\\
TC-SQL&70.9&[63.7, 68.4]&7&1-27\\
Infly-RL-SQL-32B&70.1&[63.0, 67.8]&8&1-29\\
Queryosity&69.4&[62.5, 67.3]&9&1-32\\
OpenSearch-SQL-v2 + GPT-4o&69.3&[62.4, 67.2]&10&1-32\\
GenaSQL&69.2&[62.4, 67.2]&11&1-33\\
OmniSQL-32B&69.2&[62.4, 67.1]&12&1-33\\
OmniSQL-7B&69.0&[62.2, 67.0]&13&1-33\\
PB-SQL + GPT-4o&68.6&[61.9, 66.7]&14&2-34\\
PURPLE + RED + GPT-4o&68.1&[61.5, 66.3]&15&2-34\\
Arcwise + GPT-4o&68.0&[61.4, 66.2]&16&2-34\\
Distillery + GPT-4o&67.2&[60.8, 65.6]&17&3-36\\
RSL-SQL + GPT-4o&67.2&[60.8, 65.6]&18&3-36\\
XiYanSQL-QwenCoder-32B&67.0&[60.6, 65.5]&19&4-36\\
RECAP + Gemini&67.0&[60.6, 65.4]&20&4-36\\
GSR&66.9&[60.5, 65.4]&21&4-36\\
MSL-SQL + DeepSeek-V2.5&66.8&[60.5, 65.3]&22&4-36\\
AskData + GPT-4o&65.9&[59.8, 64.6]&23&7-37\\
E-SQL + GPT-4o&65.6&[59.5, 64.4]&24&7-37\\
ByteBrain&65.5&[59.4, 64.3]&25&7-37\\
CHESS&65.0&[59.1, 63.9]&26&7-37\\
SCL-SQL&64.7&[58.9, 63.7]&27&7-39\\
EBA-SQL + GPT-4&64.6&[58.8, 63.6]&28&8-39\\
OeSQL-0.1-Qe-32B&64.6&[58.8, 63.6]&29&8-39\\
RSL-SQL + DeepSeek-v2&63.6&[58.0, 62.8]&30&9-42\\
Command-A&63.5&[57.9, 62.8]&31&9-42\\
MCS-SQL + GPT-4&63.4&[57.8, 62.7]&32&9-42\\
PURPLE + GPT-4o&63.0&[57.5, 62.4]&33&11-42\\
GRA-SQL&62.6&[57.2, 62.1]&34&14-44\\
E-SQL + GPT-4o mini&61.6&[56.4, 61.4]&35&17-46\\
OpenSearch-SQL-v1 + GPT-4&61.3&[56.2, 61.2]&36&17-46\\
Dubo-SQL-v1&59.7&[55.0, 59.9]&37&23-49\\
SuperSQL&58.5&[54.0, 59.0]&38&27-49\\
SFT CodeS-15B&58.5&[54.0, 59.0]&39&27-49\\
Chat2Query (GPT-4 + data entity modeling)&58.1&[53.8, 58.7]&40&30-50\\
MAC-SQL + GPT-4&57.6&[53.3, 58.3]&41&30-50\\
SFT CodeS-7B&57.2&[53.0, 58.0]&42&30-51\\
TA-SQL + GPT-4&56.2&[52.3, 57.2]&43&34-51\\
DeepSeek&56.1&[52.2, 57.2]&44&34-51\\
DTS-SQL + DeepSeek-7B&55.8&[52.0, 56.9]&45&35-51\\
SEE&55.5&[51.7, 56.7]&46&35-51\\
DAIL-SQL + GPT-4&54.8&[51.2, 56.1]&47&37-51\\
Interactive-T2S&54.6&[51.0, 56.0]&48&37-51\\
Mistral&53.5&[50.2, 55.2]&49&37-51\\
ExSL + granite-20b-code&51.7&[48.8, 53.8]&50&40-52\\
DIN-SQL + GPT-4&50.7&[48.0, 53.1]&51&42-52\\
GPT-4&46.4&[44.7, 49.7]&52&50-53\\
Claude-2&42.7&[41.9, 46.9]&53&52-54\\
Open-SQL&37.7&[38.1, 43.0]&54&53-54\\
Palm-2&27.4&[30.3, 35.0]&55&55-58\\
ChatGPT + CoT&25.9&[29.2, 33.8]&56&55-58\\
Codex&25.4&[28.8, 33.5]&57&55-58\\
ChatGPT&24.1&[27.8, 32.4]&58&55-58\\
T5-3B&10.4&[17.6, 21.6]&59&59-61\\
T5-Large&9.7&[17.1, 21.1]&60&59-61\\
T5-Base&6.3&[14.6, 18.4]&61&59-61\\
        \bottomrule
    \end{longtblr}
    }