\section{\name: Agentic Benchmark Checklist} \label{sec:checklist}

In this section, we formulate our assessment framework into an actionable 
checklist (\name). We present the checklist items in terms of task validity, 
outcome validity, and benchmark reporting.

\subsection{Assessing Task Validity} \label{subsec:eval-impl}
\begin{figure}
    \centering
    % \includegraphics[width=0.9\linewidth,trim={{0.05\linewidth} {0.28\linewidth} {0.05\linewidth} {0.05\linewidth}},clip]{figures/implementation-checklist.pdf}
    \includegraphics[width=\linewidth]{figures/checklist-challenge.pdf}
    \caption{Checks in \name to assess the task validity of an agentic benchmark.}
    \label{fig:impl}
\end{figure}
We propose guidelines for ensuring task validity. These checks uncover design or 
implementation flaws that can create shortcuts, which causes false positive 
evaluation results, or lead to impossible tasks, which causes false negative 
evaluation results.

\minihead{Tool} 
External tools and functions can significantly extend the capabilities of AI 
agents. Existing benchmarks provide two types of tools: self-hosted tools (\eg, 
Python, command-line tools) and API-based tools (\eg, web services). For 
self-hosted tools, it is essential to explicitly specify the correct tool or 
package versions in the prompt (\checkitem{T.1}). In terms of API-based tools, ensuring service 
availability and managing rate limits is crucial (\checkitem{T.2}). If API interruptions occur, we 
recommend detecting them and terminating the evaluation to keep benchmark users 
informed (\checkitem{T.3}).

\minihead{Environment}
Agentic benchmarks often need a sandbox environment to simulate real-world 
scenarios. Implementing and maintaining such environments can be challenging, 
especially with complex task formulations. First, to ensure the independence of 
tasks, we need to ensure that any legacy data and states are fully cleaned up 
before starting a new task (\checkitem{T.4}). For example, \kernelbench failed to remove ground 
truth answers from GPU memory, allowing agents to obtain the correct result 
through out-of-bounds memory access \cite{lange2025ai}. Furthermore, to avoid 
cheating by peeking at ground truth, it is important to fully isolate agents 
from the ground truth results (\checkitem{T.5}). Finally, the environment setup should be fully 
reproducible and frozen at the time of benchmark release (\checkitem{T.6}). Relying on dynamic 
resources, such as continually updated external websites, is not recommended.

\minihead{Implementation}
Even with a robust setup of tools and environments, subtle implementation 
vulnerabilities can also result in shortcuts or impossible tasks. Therefore, 
we recommend verifying the correctness of ground truth annotation and the 
task setup (\checkitem{T.7-8}). Providing an automatic oracle solver can help demonstrate the 
correctness of the task configuration (\checkitem{T.9}). Additionally, as demonstrated in 
\taubench \cite{yaotau}, inspecting outliers in pilot experiments is crucial for
identifying implementation bugs (\checkitem{T.10}). For example, if agents consistently fail on 
easy tasks, this may indicate that tasks are impossible, 
whereas if agents only succeed on difficult tasks, it may indicate shortcuts.

\subsection{Assessing Outcome Validity} \label{subsec:eval-design}

In this part of the assessment, we propose practical checks for ensuring the 
outcome validity of an agentic benchmark (Figure \ref{fig:design}). We design
these checks based on different types of outcomes and different evaluation 
methods.

\minihead{Information Acquisition}
To evaluate the capability of AI agents to search, retrieve, integrate, and 
summarize information, agentic benchmarks formulate tasks as information 
acquisition queries \cite{yaotau,he2024webvoyager,zhang2024cybench,zhouwebarena}. 
Depending on task requirements, benchmarks use various schemes for 
evaluating agents' textual responses, including whole string matching \cite{zhang2024cybench}, 
substring matching \cite{yaotau,zhouwebarena}, and LLM-as-a-judge \cite{zhouwebarena,he2024webvoyager}.
\begin{enumerate}[leftmargin=*]
    \item \textit{Whole String Matching} directly compares the agent's response and 
    the ground truth. When annotating ground truth, it is important to consider 
    semantically equivalent expressions (\checkitem{O.a.1}) or redundant 
    words (\checkitem{O.a.2}).\footnote[3]{In practice, users often specify format requirements for AI agents, 
    which narrows the scope of alternative expressions of the ground truth. 
    Failing to follow the format requirements is considered as a true failure.}
    \item \textit{Substring Matching} evaluates whether the agent's response contains the 
    ground truth. In addition to equivalent expressions, it should 
    handle negation modifiers (\checkitem{O.b.1}), such as ``not'' and ``negative.'' We also 
    recommend formulating tasks carefully to prevent success by 
    listing all possible answers (\checkitem{{O.b.2}}) or guessing (\checkitem{O.b.3}).
    \item \textit{LLM-as-a-Judge} uses LLMs to emulate human annotators 
    \cite{zheng2023judging,zeng2023evaluating,bavaresco2024llms,li2024leveraging,zhuge2024agent}.
    Previous studies have shown that the accuracy of LLM annotations varies 
    across domains \cite{ziems2024can}. We recommend conducting pilot 
    experiments to assess the accuracy and self-consistency of LLM judges 
    (\checkitem{O.c.1}).
\end{enumerate}

\begin{figure}[t]
    \centering
    \includegraphics[width=\linewidth]{figures/checklist-outcome.pdf}
    \caption{Checks in \name to assess the outcome validity of an agentic benchmark. 
    We group items by the types of the outcome and the methods of evaluation.}
    \label{fig:design}
\end{figure}

\minihead{Code Generation}
Existing agentic benchmarks evaluate the capability of AI agents to write code 
\cite{li2023can,jimenez2024swe,miserendino2025swe,ouyang2025kernelbench}. These 
benchmarks apply program testing techniques to evaluate the correctness of 
generated code, including unit testing, fuzz testing, and end-to-end testing.
\begin{enumerate}[leftmargin=*]
    \item \textit{Unit Testing} designs test cases for individual functions or
    classes \cite{runeson2006survey}. However, poorly constructed unit tests can 
    lead to both false positives and false negatives \cite{stroebl2024inference,
    utboost}. Therefore, we recommend manually verifying the correctness and 
    quality of test cases (\checkitem{O.d.1}) \cite{swebench-verified}, and 
    providing quality guarantees using objective metrics (\checkitem{O.d.2}) 
    such as coverage \cite{zhu1997software} and cyclomatic complexity 
    \cite{watson1996structured}.
    \item \textit{Fuzz Testing} evaluates generated code by running it against 
    a ground-truth implementation on automatically generated inputs 
    \cite{zhu2022fuzzing}. We should tailor the input generator to the target
    program, covering different data values, types, memory layouts, and edge 
    cases (\checkitem{O.e.1-2}). Moreover, the inputs must affect the 
    output (\checkitem{O.e.3})---e.g., random negatives reveal nothing about \texttt{relu(x)} \cite{lange2025ai}.
    \item \textit{End-to-end (E2E) Testing} simulates complete user workflows, 
    providing comprehensive testing of system functionality \cite{tsai2001end,
    leotta2023challenges}. In addition to ensuring the general quality of test 
    cases, it should also cover all possible branches of user 
    workflows (\checkitem{O.f.1}). Because of their complexity, E2E tests require extra safeguards 
    to eliminate non-determinism and ensure repeatable results (\checkitem{O.f.2})
    \cite{parry2021survey}.
\end{enumerate}

\minihead{State Modification}
Agentic benchmarks challenge agents to manipulate environment states, such as 
booking flight tickets \cite{yaotau} and editing websites \cite{xie2024osworld}. 
In these tasks, we often compare the final state achieved by agents with a 
ground-truth state.

We identify three key checks for rigorous state matching. First, 
ground truth states should include all possible outcomes achievable through 
successful task resolution (\checkitem{O.g.1}). For example, when we challenge 
agents to attack a website, we should evaluate all possible attack outcomes 
\cite{zhu2025cve}. Second, the state space should contain both relevant and 
irrelevant states (\checkitem{O.g.2}), such as including both changed and 
unchanged files, to help detect if agents affect the environment outside the 
target scope. Finally, the state space should be complex enough 
(\checkitem{O.g.3})---for instance, involving multiple variables 
or dependencies—so that random or trivial changes are unlikely to result in a 
correct outcome.

\minihead{Multistep Reasoning}
Agentic benchmarks evaluates multistep reasoning capabilities of AI agents 
\cite{mialon2023gaia,chan2024mle,glazer2024frontiermath,lightman2023let}. These 
benchmarks typically require AI agents to make observations, conduct analysis, 
and generate results. We summarize two common approaches for evaluating these 
tasks:
\begin{enumerate}[leftmargin=*]
    \item \textit{Answer Matching} parses the agents' output and
    then compares the parsed result with ground truth.
    We find that parsers in existing benchmarks may make implicit assumption 
    about the agent's output (\checkitem{O.h.1}). For example, the MATH dataset 
    assumes the answer of the agent starts with ``Answer:'' \cite{lightman2023let}. Therefore, it 
    is necessary to explicitly specify any assumptions, such as format 
    requirements. Additionally, to ensure that a single final answer reflects a 
    genuine reasoning process, we recommend designing tasks in a way that 
    avoid success by guessing (\checkitem{O.h.2}) \cite{glazer2024frontiermath}.
    \item \textit{Quality Measure} evaluates agent using customized 
    metrics against a baseline when ground truth is impossible to achieve (\eg, 
    ground-truth predictions in an ML engineering task \cite{chan2024mle}).
    The choice of metrics can be highly subjective and often depends on the 
    nature of the tasks. To avoid metric hacking \cite{head2015extent}---achieving
    high metrics without resolving tasks, we recommend ensuring that the 
    selected metrics are strongly correlated with the reasoning process (\checkitem{O.i.1}).
\end{enumerate}

\subsection{Assessing the Benchmark Reporting} \label{subsec:eval-report}
\begin{figure}
    \centering
    % \includegraphics[width=0.75\linewidth,trim={{0.05\linewidth} {0.28\linewidth} {0.05\linewidth} {0.05\linewidth}},clip]{figures/reporting-checklist.pdf}
    \includegraphics[width=\linewidth]{figures/checklist-reporting.pdf}
    \caption{Checks in \name to assess the benchmark reporting.}
    \label{fig:report}
\end{figure}

Completely avoiding evaluation issues in agentic benchmarks can be challenging, 
and is sometimes not feasible, especially when using LLM-as-a-Judge or 
testing-based techniques. In such cases, it is particularly important for 
benchmark developers to be transparent and clearly communicate the impact of
these limitations  (Figure \ref{fig:report}).

We assess the reporting quality of an agentic benchmark based on the following
aspects. In Appendix~\ref{sec:app-example}, we use \birdbench as an example to 
demonstrate the high-quality benchmark reporting.
\begin{enumerate}[leftmargin=*]
    \item \textit{Transparency and Validity}. We encourage open-sourcing both 
    the datasets and evaluation harness (\checkitem{R.1-2}) while including 
    measures to prevent data contamination (\checkitem{R.3-4}). We also 
    recommend clearly specifying the capabilities to evaluate
    and articulating construct validity \cite{reuelbetterbench} (\checkitem{R.5-6}).
    \item \textit{Mitigation}. When validity limitations are unavoidable, it is 
    important to document mitigation efforts (\checkitem{R.7}) and provide both qualitative and 
    quantitative evidence regarding the impact of those limitations (\checkitem{R.8-9}). In 
    resource-constraint scenarios, we recommend using sampling and uncertainty 
    quantification techniques (\eg, Cramer's Theorem \cite{dorner2024don}) to 
    estimate the impact of unavoidable flaws, such as the noise of ground truth.
    \item \textit{Result Interpretation}. We recommend reporting benchmark 
    results rigorously, including measures of statistical significance (\checkitem{R.10}), clear 
    interpretation guidelines (\checkitem{R.11}), and appropriate baseline comparisons (\checkitem{R.12-13}).
\end{enumerate}
