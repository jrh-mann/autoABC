\section{Introduction}

AI agents that integrate machine learning models with tools, memory, and
knowledge are emerging with the capability to solve complex problems
\cite{yang2024swe,yaoreact,chen2023autoagents,he2024webvoyager,
shinn2023reflexion,lin2023swiftsage,tang2024worldcoder}. To evaluate AI agents, 
researchers and practitioners have built \textit{agentic benchmarks} with 
realistic tasks to track progress and assist decision-making \cite{zhouwebarena,
yaotau,jimenez2024swe,xie2024osworld,chan2024mle,li2023can,glazer2024frontiermath,
zhang2024cybench,ouyang2025kernelbench,mialon2023gaia}.
AI agents have exhibited impressive performance on these benchmarks. For example, 
a GPT-4o-based agent resolves 35\% of tasks on \taubench-Airline, a benchmark 
for tool-agent-user interaction \cite{yaotau}. As agentic benchmarks become 
increasingly impactful in academia and industry, it is crucial to ensure these
numbers can be trusted.

Agentic benchmarks differ fundamentally from traditional AI benchmarks. 
Multiple-choice datasets (\eg, ImageNet \cite{deng2009imagenet} and MMLU 
\cite{hendrycks2020measuring}) evaluate models by their accuracy on categorical labels, 
while text-generation benchmarks rely on automatic metrics (\eg, BLEU 
\cite{papineni2002bleu}). By contrast, success is defined by 
completing end-to-end tasks in agentic settings. Therefore, an agent may perform 
coherent reasoning, write code, and execute commands to produce a final outcome. 
Subsequently, the performance of the agent is determined by comparing its final 
outcome with a ground-truth outcome, using various methods, such as program 
testing and string matching \cite{zhouwebarena,yaotau,jimenez2024swe,
xie2024osworld,chan2024mle,li2023can,zhang2024cybench,ouyang2025kernelbench,
mialon2023gaia}. 

Unfortunately, many existing outcome-based evaluation methods of agentic 
benchmarks introduce issues that can cause under- or overestimation of agent capabilities by 
up to 100\% in relative terms, compromising the validity of their findings 
\cite{utboost,lange2025ai,wretblad2024understanding,pourreza2023evaluating,metr-kernel}. For 
example, \swebench-Verified challenges an agent to resolve GitHub issues, and 
considers the agent successful if the patch it generates passes manually vetted 
unit tests \cite{swebench-verified}. However, recent work has shown that passing 
these tests does not necessarily indicate that the issue is resolved, because unit 
tests can fail to capture important edge cases. Consequently, 24\% of the top 50
leaderboard positions are incorrect \cite{utboost,swe-leaderboard}. In addition, 
we find that in \taubench, a trivial agent that returns empty responses is 
considered successful on intentionally impossible tasks (\eg, changing a 
non-refundable ticket). This trivial agent achieves a 38\% success rate and 
outperforms a GPT-4o-based agent \cite{yaotau}.

Although issues in evaluation rigor can significantly skew evaluation results, 
they are still frequently overlooked in the current development, deployment, and 
analysis of agentic benchmarks. To better understand this problem, we analyzed 
prior work on agentic benchmark pitfalls  \cite{utboost,lange2025ai,
wretblad2024understanding,pourreza2023evaluating,metr-kernel} and 17 widely used
agentic benchmarks (Table \ref{tab:all-benchmarks}), such as
\swebench-Verified \cite{swebench-verified}, \gaia \cite{mialon2023gaia}, 
\taubench \cite{yaotau}, and \webarena \cite{zhouwebarena}. Combining insights from the 
literature with our own experience in developing benchmarks, we identified two 
major conditions of the validity of benchmark results:
\begin{itemize}[leftmargin=*]
    \item \textit{Outcome validity}: the evaluation result (\eg, tests or 
    checks) truly indicates task success. \swebench-Verified fails here 
    because an incorrect patch can still pass the test suite.
    \item \textit{Task validity}: a task should be solvable if and only if the 
    agent possesses the target capability. \revise{Issues in task design or 
    implementation often breaks task validity.} For example, \taubench allows a 
    trivial agent to pass 38\% of tasks without knowledge of airline-ticketing rules.
\end{itemize}

%% introduce our work
Following prior work on analyzing AI and code benchmarks \cite{reuelbetterbench,
cao2025should}, we formulate our insights into an \textbf{A}gentic 
\textbf{B}enchmark \textbf{C}hecklist (\name) to assist benchmark developers and 
users in critically designing and assessing agentic benchmarks. Using \name, we 
assessed ten popular agentic benchmarks that span the full range of agent 
capabilities, resulting in seven benchmarks with flaws in outcome validity, seven 
with issues in task validity, and all with limitations in the result 
reporting. In addition to the issues found in \taubench-Airline, some other 
example issues we found are: 
(1) an agent can score 100\% on \swelancer \cite{miserendino2025swe} without resolving any tasks; 
(2) \kernelbench \cite{ouyang2025kernelbench} overestimates agents' capabilities in generating correct 
kernel functions by 31\% in absolute terms due to incomprehensive fuzz testing; 
(3) \webarena \cite{zhouwebarena} overestimates performance of agents by 5.2\% due to various issues in its string matching.
To demonstrate ABC’s practical value, we applied it to improve \cvebench, a 
complex, representative cybersecurity benchmark
\cite{zhu2025cve}. \name reduced performance overestimation in \cvebench by 
33\% in absolute terms, as confirmed by cybersecurity experts.

%% contributions 
We summarize our contributions as follows:
\begin{enumerate}[leftmargin=*]
    \item We identified two significant threats in the evaluation rigor of agentic 
    benchmarks: outcome validity and task validity.
    \item We developed an actionable checklist, \name, to critically assess
    existing agentic benchmarks and to establish best practices for future 
    development.
    \item We applied \name to assess ten widely used agentic benchmarks and 
    identified new evaluation issues that cause estimation errors of agents' 
    performance by up to 100\% in relative terms.
    \item We provided a case study of using \name to improve an agentic benchmark
    during development.
\end{enumerate}

% We structure the rest of this paper as follows. In Section~\ref{sec:rel-word}, 
% we review the existing literature and efforts in benchmarking AI agents and 
% analyzing agentic benchmarks. We discuss the design of our study in 
% Section~\ref{sec:method}. Next, Section~\ref{sec:checklist} introduces the 
% actionable checklists for agentic benchmarks. Finally, we present our 
% qualitative and quantitative findings in Section~\ref{sec:results}.



% \begin{table}[t]
%     \centering
%     \caption{Difference between Agentic Benchmarks and Multiple-Choice Benchmarks.}
%     \label{tab:comp}
%     \begin{tabular}{c|c|c}
%         \toprule
%          & Agentic Benchmark & Multiple-Choice Benchmark \\
%         \midrule
%         Evaluated capability & High-level, multidimensional & Fundamental, one-dimension \\
%         Data annotation & More complex & Less complex \\
%         \# Reasoning steps & $>1$ & $\sim 1$ \\
%         \# Ways to complete a task & > 1 & = 1 \\
%         $\mathbb{P}[\mathrm{Success\ by\ guess}]$ & $\sim 0$ & $=\frac{1}{\mathrm{\# choices}}$ \\
%         \makecell{Evaluation method} & Task-specific & Direct match \\
%         \bottomrule
%     \end{tabular}
% \end{table}
